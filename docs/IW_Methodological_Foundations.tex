\documentclass[12pt,a4paper]{article}

\usepackage[utf8]{inputenc}
\usepackage[T1]{fontenc}
\usepackage[brazil]{babel}
\usepackage{geometry}
\usepackage{setspace}
\usepackage{hyperref}

\geometry{margin=2.5cm}
\onehalfspacing

\title{Document-Driven Development, Domain-Driven Design\\
e o Papel da Narrativa no IdeaWalker}
\author{Projeto IdeaWalker}
\date{\today}

\begin{document}

\maketitle

\section{Introdução}

O projeto \textit{IdeaWalker (IW)} nasce da necessidade de apoiar processos
de investigação, escrita e análise científica em contextos fortemente
documentais, exploratórios e humano--IA.
Desde sua concepção, tornou-se evidente que abordagens tradicionais de
engenharia de software --- centradas em bancos de dados, estados internos
ou modelos rígidos --- não eram adequadas para a natureza epistemológica
do problema.

Este documento formaliza as decisões metodológicas que orientam o IW,
em especial:
\begin{itemize}
    \item a adoção do \textit{Document-Driven Development} como fundamento,
    \item o uso do \textit{Domain-Driven Design} como mecanismo de governança,
    \item e o papel da \textbf{Narrativa} como contexto de aplicação das
    análises discursivas.
\end{itemize}

\section{Document-Driven Development como Fundamento}

No IdeaWalker, documentos não são meros artefatos auxiliares,
mas constituem a \textbf{fonte primária de verdade} do sistema.

Notas em Markdown e dados estruturados em JSON representam estados
persistentes, auditáveis e interoperáveis do conhecimento produzido.
Essa escolha garante:
\begin{itemize}
    \item transparência epistemológica,
    \item longevidade dos dados independentemente do software,
    \item interação direta entre humanos e sistemas de IA,
    \item rastreabilidade de processos cognitivos.
\end{itemize}

Nesse sentido, o \textit{Document-Driven Development} governa
\textbf{como o conhecimento existe} no sistema.

\section{Domain-Driven Design como Governança Semântica}

À medida que o sistema cresce, surge o risco de entropia conceitual:
documentos passam a acumular significados ambíguos, interpretações
inconsistentes e inferências indevidas.

O \textit{Domain-Driven Design (DDD)} é introduzido não como dogma técnico,
mas como mecanismo de \textbf{governança semântica}.
Seu papel no IW é:
\begin{itemize}
    \item definir vocabulário ubíquo,
    \item explicitar limites de interpretação,
    \item proteger invariantes epistemológicas,
    \item impedir que decisões conceituais sejam tomadas implicitamente.
\end{itemize}

No IW, o DDD não modela dados,
mas o \textbf{direito de interpretar dados}.

\section{Separação entre Contexto Epistemológico e Contexto Computacional}

O projeto reconhece explicitamente dois contextos distintos e hierárquicos:

\subsection{Contexto Epistemológico}

Responsável por definir:
\begin{itemize}
    \item fundamentos teóricos,
    \item pressupostos metodológicos,
    \item limites éticos e científicos,
    \item aquilo que pode ou não ser inferido.
\end{itemize}

Este contexto é expresso primariamente em documentos
e não possui caráter executável.

\subsection{Contexto Computacional}

Responsável por:
\begin{itemize}
    \item implementar estruturas de domínio,
    \item garantir invariantes em tempo de execução,
    \item orquestrar fluxos de processamento,
    \item fazer cumprir, em código, as decisões epistemológicas.
\end{itemize}

O contexto computacional não redefine teoria;
ele a \textbf{operacionaliza e protege}.

\section{Narrativa como Contexto de Aplicação da Análise Discursiva}

No IdeaWalker, a narrativa não é tratada como simples conteúdo textual,
mas como \textbf{infraestrutura pragmática de sentido}.

A narrativa define o \textit{Contexto de Aplicação} da análise discursiva,
estabelecendo:
\begin{itemize}
    \item intencionalidade do texto,
    \item enquadramento temporal,
    \item estatuto dos enunciados,
    \item limites interpretativos válidos.
\end{itemize}

Assim, o mesmo elemento sintático pode assumir significados distintos
conforme a narrativa na qual está inserido.
A narrativa não decide resultados nem produz inferências,
mas \textbf{condiciona as regras sob as quais a interpretação é válida}.

\section{Limites Epistemológicos da Inferência}

Um princípio central do IW é a distinção entre:
\begin{itemize}
    \item causalidade inferida pelo sistema,
    \item e causalidade declarada no discurso analisado.
\end{itemize}

O sistema observa, registra e estrutura como relações causais,
motivações ou julgamentos são \textbf{enunciados},
sem transformá-los automaticamente em verdades ontológicas.

\section{Conclusão}

O IdeaWalker assume explicitamente que:
\begin{itemize}
    \item é \textbf{Document-Driven por natureza},
    \item é \textbf{governado por Domain-Driven Design},
    \item e utiliza a \textbf{Narrativa} como chave para a aplicação
    responsável de análises discursivas.
\end{itemize}

Essa combinação permite preservar plasticidade cognitiva,
rigor metodológico e coerência técnica,
sem sacrificar a natureza exploratória do trabalho científico.

\end{document}
