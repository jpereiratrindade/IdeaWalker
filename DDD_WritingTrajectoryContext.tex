\documentclass[12pt,a4paper]{article}

% -------------------------------------------------
% Pacotes básicos
% -------------------------------------------------
\usepackage[T1]{fontenc}
\usepackage[utf8]{inputenc}
\usepackage[brazil]{babel}
\usepackage{setspace}
\usepackage{geometry}
\usepackage{hyperref}
\usepackage{enumitem}
\usepackage{csquotes}

% -------------------------------------------------
% Configuração de página
% -------------------------------------------------
\geometry{
	top=3cm,
	bottom=2.5cm,
	left=3cm,
	right=2.5cm
}
\onehalfspacing
\setlength{\parindent}{1.2cm}
\setlength{\parskip}{0.2cm}

% -------------------------------------------------
% Título
% -------------------------------------------------
\title{\textbf{WritingTrajectoryContext (WTC)}\\
\large Domain-Driven Design para Trajetórias de Escrita com Autoria Cognitiva, Rastreabilidade e Defesa}
\author{}
\date{}

\begin{document}
\maketitle
\tableofcontents*
\newpage

% =================================================
\section{Purpose and Epistemic Contract}

\subsection{Purpose}
Modelar a escrita como uma trajetória versionada de decisões, onde o valor está no processo:
intenção $\rightarrow$ estrutura $\rightarrow$ rascunhos $\rightarrow$ revisões $\rightarrow$ defesa.

\subsection{Epistemic Contract}
\begin{itemize}[leftmargin=*]
  \item A IA pode participar como \textbf{instrumento} (sugestão, revisão, crítica), nunca como \textbf{autor} do compromisso argumentativo.
  \item A autoria é definida pela \textbf{trajetória decisional humana} e sua \textbf{defensabilidade}.
  \item O contexto registra \textbf{o que mudou}, \textbf{por que mudou} e \textbf{quais alternativas foram consideradas}.
  \item Detecção de ``IA vs humano'' é \textbf{fora de escopo}. O foco é \textbf{controle epistêmico e rastreabilidade}.
\end{itemize}

% =================================================
\section{Bounded Context}

\subsection{Nome}
WritingTrajectoryContext (WTC)

\subsection{Responsabilidades}
\begin{itemize}[leftmargin=*]
  \item Gerenciar trajetórias de escrita (do outline ao texto final).
  \item Versionar segmentos e registrar decisões editoriais.
  \item Manter ligações semânticas com notas/artefatos (backlinks, fontes, evidências).
  \item Produzir material para \textbf{defesa} (auto-arguição / perguntas críticas).
\end{itemize}

\subsection{Fora de Escopo}
\begin{itemize}[leftmargin=*]
  \item Geração automática de ``texto final'' como substituição do autor.
  \item Otimização para evasão de detectores.
  \item Verificação externa de fatos (pode integrar, mas não pertence ao domínio).
\end{itemize}

% =================================================
\section{Ubiquitous Language}

\begin{itemize}[leftmargin=*]
  \item \textbf{Trajectory}: sequência de estados de escrita com eventos e decisões.
  \item \textbf{Segment}: unidade mínima versionada (parágrafo, seção, bloco).
  \item \textbf{Revision}: transformação de um segmento com propósito e justificativa.
  \item \textbf{Decision}: escolha argumentativa/editorial explicitada e justificável.
  \item \textbf{Evidence Link}: referência a fonte, nota, dado, citação.
  \item \textbf{Defense Prompt}: pergunta de arguição que valida compreensão.
\end{itemize}

% =================================================
\section{Core Domain Model}

\subsection{Aggregate Root: WritingTrajectory}

\textbf{WritingTrajectory} representa um trabalho textual em evolução (ex.: relatório, capítulo, artigo).

\subsubsection{Identity}
\begin{itemize}[leftmargin=*]
  \item TrajectoryId (UUID)
\end{itemize}

\subsubsection{State (Value Object)}
\begin{itemize}[leftmargin=*]
  \item \textbf{Stage}: \{Intent, Outline, Drafting, Revising, Consolidating, ReadyForDefense, Final\}
  \item \textbf{Confidence}: 0..1 (autoavaliação do autor, opcional)
  \item \textbf{LastUpdated}
\end{itemize}

\subsubsection{Children / Components}
\begin{itemize}[leftmargin=*]
  \item WritingIntent (VO)
  \item OutlineGraph (Entity) \textit{(estrutura e ordem de segmentos)}
  \item DraftSegment (Entity) \textit{(conteúdo versionado)}
  \item RevisionDecision (Entity)
  \item EvidenceLink (VO)
  \item AIInteractionRecord (VO) \textit{(transparência e metacognição)}
  \item DefenseCard (Entity)
\end{itemize}

\subsubsection{Aggregate Invariants}
\begin{itemize}[leftmargin=*]
  \item Não existe revisão sem \textbf{Rationale} (decisão explicitada).
  \item Todo segmento publicado como ``Final'' deve ter:
  \begin{itemize}
    \item pelo menos 1 revisão humana (source=human\_reviewed)
    \item pelo menos 1 defense card associado (ou justificativa de dispensa)
  \end{itemize}
  \item Toda evidência linkada deve apontar para \textbf{um artefato rastreável} (nota, DOI, arquivo, entrevista).
  \item Stage só avança por \textbf{Domain Events} (sem pulo mágico).
\end{itemize}

% =================================================
\section{Entities and Value Objects}

\subsection{WritingIntent (Value Object)}
\begin{itemize}[leftmargin=*]
  \item Purpose: informar | argumentar | relatar | propor | instruir
  \item Audience: banca | cliente | público geral | equipe técnica
  \item CoreClaim / ResearchQuestion
  \item Constraints: tamanho, normas (ABNT), prazo, tom
\end{itemize}
Invariante: não pode ser vazio.

\subsection{DraftSegment (Entity)}
\begin{itemize}[leftmargin=*]
  \item SegmentId
  \item Title / Label
  \item Content
  \item SourceTag: human | ai\_assisted | ai\_generated | human\_reviewed
  \item Version: monotônica
  \item Timestamp
\end{itemize}

\subsection{RevisionDecision (Entity)}
\begin{itemize}[leftmargin=*]
  \item DecisionId
  \item Target: SegmentId
  \item Operation: clarify | compress | expand | reorganize | cite | remove | reframe
  \item Rationale: texto curto, obrigatório
  \item AlternativesConsidered: lista curta (opcional, mas recomendada)
\end{itemize}

\subsection{EvidenceLink (Value Object)}
\begin{itemize}[leftmargin=*]
  \item RefType: note | pdf | doi | url | interview | dataset
  \item RefId (ou caminho/identificador)
  \item ClaimAnchor: trecho/posição do segmento (range)
  \item Confidence: 0..1
\end{itemize}

\subsection{AIInteractionRecord (Value Object)}
\begin{itemize}[leftmargin=*]
  \item Tool: qwen | llama | gpt | etc.
  \item Intent: brainstorm | critique | rewrite | outline | citation\_ideas
  \item InputHash (não precisa armazenar o texto bruto sempre)
  \item OutputSummary (curto)
  \item Timestamp
\end{itemize}

\subsection{DefenseCard (Entity)}
\begin{itemize}[leftmargin=*]
  \item CardId
  \item SegmentId (ou OutlineNodeId)
  \item Prompt: pergunta de defesa
  \item ExpectedDefensePoints: bullets (opcional)
  \item Status: pending | rehearsed | passed
\end{itemize}

% =================================================
\section{Domain Events}

Eventos que \textbf{mudam estado} e registram história.

\begin{itemize}[leftmargin=*]
  \item TrajectoryCreated
  \item IntentDeclared
  \item OutlineUpdated
  \item SegmentAdded
  \item SegmentRevised
  \item DecisionRecorded
  \item EvidenceLinked
  \item DefenseCardsGenerated
  \item StageAdvanced
  \item TrajectoryFinalized
\end{itemize}

% =================================================
\section{Commands (Application Use Cases)}

\begin{itemize}[leftmargin=*]
  \item CreateTrajectory(intent)
  \item UpdateOutline(graphDelta)
  \item AddSegment(label, content, sourceTag)
  \item ReviseSegment(segmentId, newContent, operation, rationale, sourceTag)
  \item LinkEvidence(segmentId, evidenceLink)
  \item GenerateDefenseCards(scope)
  \item AdvanceStage(expectedStage)
  \item ExportArtifact(format=\{md, tex\})
\end{itemize}

Regra: todo command deve resultar em 0..N domain events.

% =================================================
\section{Domain Services}

\subsection{CoherenceLensService}
Verifica alinhamento: Intent $\leftrightarrow$ Outline $\leftrightarrow$ Segmentos.
Saída: lista de inconsistências e sugestões (não altera domínio).

\subsection{RevisionQualityService}
Classifica revisões por propósito (clareza/coerência/densidade) e detecta:
\begin{itemize}[leftmargin=*]
  \item homogeneização excessiva do estilo
  \item perda de termos técnicos essenciais
  \item cortes que removem suporte evidencial
\end{itemize}

\subsection{DefensePromptFactory}
Gera prompts de defesa com base em:
\begin{itemize}[leftmargin=*]
  \item decisões registradas
  \item pontos de tensão argumentativa
  \item trechos com baixa evidência/confiança
\end{itemize}

% =================================================
\section{Integration with IdeaWalker}

\subsection{Context Map}
\begin{itemize}[leftmargin=*]
  \item \textbf{Note/Knowledge Context} (upstream): fornece notas e backlinks como evidências.
  \item \textbf{Cognitive Dialogue Context} (collaboration): conversas podem gerar decisões e defense cards.
  \item \textbf{PersistenceService} (infrastructure): grava eventos, snapshots e exportações de modo atômico.
\end{itemize}

\subsection{Anti-Corruption Layer (ACL)}
Um adaptador protege o domínio WTC de formatos externos:
\begin{itemize}[leftmargin=*]
  \item NotesAdapter: converte notas em EvidenceLink.
  \item AIAdapter: converte interações em AIInteractionRecord (sem vazar prompts brutos se não desejado).
\end{itemize}

\subsection{Storage Strategy (Event + Snapshot Hybrid)}
\begin{itemize}[leftmargin=*]
  \item \textbf{Event Log} (fonte da verdade): \texttt{/writing/trajectories/<id>/events.ndjson}
  \item \textbf{Snapshots} (performance): \texttt{/writing/trajectories/<id>/snapshot.json}
  \item \textbf{Exports}: \texttt{/writing/trajectories/<id>/exports/<name>.tex/.md}
\end{itemize}

% =================================================
\section{Suggested Folder Layout (C++ Project)}

\begin{verbatim}
src/domain/writing/
  WritingTrajectory.hpp/.cpp
  value_objects/
    WritingIntent.hpp
    EvidenceLink.hpp
    TrajectoryStage.hpp
    AIInteractionRecord.hpp
  entities/
    DraftSegment.hpp
    RevisionDecision.hpp
    DefenseCard.hpp
  services/
    CoherenceLensService.hpp
    RevisionQualityService.hpp
    DefensePromptFactory.hpp
  events/
    WritingEvents.hpp
  repositories/
    IWritingTrajectoryRepository.hpp

src/application/writing/
  WritingTrajectoryAppService.hpp/.cpp
  ExportService.hpp/.cpp

src/infrastructure/writing/
  WritingTrajectoryRepositoryFs.hpp/.cpp
  WritingEventStoreFs.hpp/.cpp
  adapters/
    NotesAdapter.hpp/.cpp
    AIAdapter.hpp/.cpp
\end{verbatim}

% =================================================
\section{UI Integration (Panels)}

Painéis mínimos:
\begin{itemize}[leftmargin=*]
  \item \textbf{TrajectoryPanel}: lista e status das trajetórias.
  \item \textbf{OutlinePanel}: grafo/árvore da estrutura.
  \item \textbf{SegmentEditorPanel}: edição + versão + rationale.
  \item \textbf{DefensePanel}: cards de defesa e modo ensaio.
\end{itemize}

Regra UI: o usuário sempre vê \textbf{o porquê} (rationale) como cidadão de primeira classe.

% =================================================
\section{Operational Workflow}

\begin{enumerate}[leftmargin=*]
  \item CreateTrajectory + Declare Intent
  \item Construir Outline humano
  \item Rascunhar segmentos (human/ai\_assisted)
  \item Revisar com decisões (rationale obrigatório)
  \item Linkar evidências (notas, PDFs, DOI)
  \item Gerar DefenseCards
  \item Ensaiar defesa (auto-arguição)
  \item Exportar em \texttt{.tex} (ABNT) e/ou \texttt{.md}
\end{enumerate}

\end{document}
